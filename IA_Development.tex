\documentclass[conference]{IEEEtran}
\IEEEoverridecommandlockouts
% The preceding line is only needed to identify funding in the first footnote. If that is unneeded, please comment it out.
%Template version as of 6/27/2024

\usepackage{cite}
\usepackage{amsmath,amssymb,amsfonts}
\usepackage{algorithmic}
\usepackage{graphicx}
\usepackage{textcomp}
\usepackage{xcolor}
\def\BibTeX{{\rm B\kern-.05em{\sc i\kern-.025em b}\kern-.08em
    T\kern-.1667em\lower.7ex\hbox{E}\kern-.125emX}}
\begin{document}

\title{Formulación del Problema de Investigación:\\
Inteligencia Artificial Aplicada al Desarrollo de Software\\
en Contextos Empresariales}

\author{\IEEEauthorblockN{1\textsuperscript{st} Mesias Orlando Mariscal Oña}
\IEEEauthorblockA{\textit{dept. name of organization (of Aff.)} \\
\textit{name of organization (of Aff.)}\\
City, Country \\
email address or ORCID}
\and
\IEEEauthorblockN{2\textsuperscript{nd} Denise Noemi Rea Diaz}
\IEEEauthorblockA{\textit{dept. name of organization (of Aff.)} \\
\textit{name of organization (of Aff.)}\\
City, Country \\
email address or ORCID}
\and
\IEEEauthorblockN{3\textsuperscript{rd} Julio Enrique Viche Castillo}
\IEEEauthorblockA{\textit{dept. name of organization (of Aff.)} \\
\textit{name of organization (of Aff.)}\\
City, Country \\
email address or ORCID}
}

\maketitle

\begin{abstract}
La adopción de tecnologías de Inteligencia Artificial (IA) en el desarrollo de software empresarial ha evolucionado significativamente en los últimos años. Aunque existen estudios sobre viabilidad técnica de Machine Learning y Deep Learning en tareas específicas de ingeniería de software, existe una brecha crítica en la literatura respecto a cómo estas tecnologías impactan la transformación organizacional, la evolución de competencias profesionales y la adopción sostenida en contextos empresariales reales. Este documento presenta la formulación del problema de investigación, la identificación del vacío científico y la propuesta de preguntas y objetivos de investigación que guiarán un estudio sobre el impacto de la IA en la ingeniería de software empresarial.
\end{abstract}

\begin{IEEEkeywords}
Inteligencia Artificial, Ingeniería de Software, Transformación Organizacional, Competencias Profesionales, Machine Learning, Generative AI, Adopción Empresarial, Desarrollo de Software.
\end{IEEEkeywords}

\section{Introducción}

La Inteligencia Artificial ha transformado el desarrollo de software empresarial en los últimos años. Herramientas como ChatGPT y GitHub Copilot están automatizando tareas críticas: generación de código, detección de defectos y estimación de esfuerzo. Sin embargo, existe una brecha entre la viabilidad técnica demostrada en la literatura académica y la adopción real en organizaciones empresariales.

Para profesionales en formación en Ingeniería de Software surgen interrogantes críticas: ¿Qué competencias permanecerán relevantes con IA? ¿Cómo transformarán estos cambios los perfiles profesionales? ¿Cómo pueden las organizaciones adoptar IA de manera confiable?

Este documento presenta la \textbf{Fase 1} de un proyecto orientado a formular un problema de investigación riguroso. A través del análisis de 10 artículos científicos recientes (2021-2025), se identifican vacíos en la literatura que guíen futuras investigaciones.


\section{Fase 1: Identificación y Contextualización del Problema}

\subsection{Metodología de Búsqueda y Cadenas de Búsqueda}

La identificación de artículos científicos se realizó mediante una búsqueda sistemática en múltiples bases de datos y plataformas bibliográficas durante noviembre de 2025.

\subsubsection{Bases de datos consultadas:}
\begin{itemize}
\item \textbf{OpenAlex API:} Base de datos bibliográfica integral que indexa publicaciones de ACM, IEEE, arXiv y Scopus (8 artículos recuperados).
\item \textbf{CrossRef API:} Registro oficial de DOIs incluyendo IEEE, ACM, Springer (2 artículos recuperados).
\item \textbf{Semantic Scholar API:} Análisis de contenido basado en ML para identificación de relevancia (consultas iniciales).
\item \textbf{Zotero:} Herramienta de gestión bibliográfica para organización y exportación de referencias.
\item \textbf{Research Rabbit:} Plataforma visual para descubrimiento de papers conectados y análisis de redes de citaciones.
\end{itemize}

\subsubsection{Cadenas de búsqueda utilizadas:}

Las siguientes cadenas de búsqueda fueron aplicadas sistemáticamente:

\begin{enumerate}
\item \textbf{Búsqueda Principal (OpenAlex):}
\begin{quote}
\small{``(machine learning OR deep learning OR artificial intelligence OR generative AI) AND (software development OR software engineering) AND (enterprise OR industry OR business) AND published\_date:[2021-01-01 TO 2025-12-31]''}
\end{quote}

\item \textbf{Búsqueda Alternativa (OpenAlex):}
\begin{quote}
\small{``(AI adoption OR digital transformation OR technology acceptance) AND software engineering AND enterprise AND published\_date:[2021-01-01 TO 2025-12-31]''}
\end{quote}

\item \textbf{Búsqueda Específica (CrossRef):}
\begin{quote}
\small{``generative AI software development chatgpt copilot 2023 2024 2025''}
\end{quote}
\end{enumerate}

\subsubsection{Criterios de inclusión y exclusión:}

\textbf{Criterios de Inclusión:}
\begin{itemize}
\item Publicado entre 2021-2025 (últimos 5 años).
\item Idioma: inglés o español.
\item Temática: IA aplicada al desarrollo de software, ingeniería de software con ML/DL, o generative AI en contexto empresarial.
\item Tipo de documento: artículos peer-reviewed, case studies, surveys sistemáticos o revisiones de literatura.
\item Contiene validación empírica o análisis en contexto empresarial.
\end{itemize}

\textbf{Criterios de Exclusión:}
\begin{itemize}
\item Trabajos no relacionados con desarrollo de software.
\item Estudios teóricos sin validación empírica.
\item Artículos sin DOI verificable.
\item Papers sobre IA general sin conexión con ingeniería de software.
\end{itemize}

\subsection{1.2. Artículos Analizados (2021-2025)}

Se analizaron 10 artículos científicos que representan el estado actual de la investigación en IA aplicada al desarrollo de software empresarial:

\begin{enumerate}
\item Mustyala et al. (2025) - Case Studies: Machine Learning Approaches for Effort Estimation
\item Rajbhoj et al. (2024) - Accelerating Software Development Using Generative AI: ChatGPT Case Study
\item Russo (2024) - Navigating the Complexity of Generative AI Adoption
\item Bahi et al. (2024) - Generative AI for Advancing Agile Software Development
\item Ebert \& Louridas (2023) - Generative AI for Software Practitioners
\item Wang et al. (2023) - Machine/Deep Learning for SE: A Systematic Literature Review
\item Sofian et al. (2022) - AI Techniques in SE: Systematic Mapping
\item Liu et al. (2021) - Reproducibility and Replicability of Deep Learning in SE
\item Hutchinson et al. (2021) - Towards Accountability for ML Datasets
\item Yang et al. (2022) - A Survey on Deep Learning for Software Engineering
\end{enumerate}

\subsection{1.3. Hallazgos Clave de los 10 Artículos}

\subsubsection{Tendencias Identificadas (2021-2025):}

\begin{enumerate}
\item \textbf{Transición de ML Tradicional a Generative AI:} Estudios iniciales (2021-2022) enfatizaban Machine Learning y Deep Learning en tareas específicas. A partir de 2023-2024, giro pronunciado hacia Generative AI (LLMs), ampliando aplicaciones pero introduciendo incertidumbres sobre reproducibilidad.

\item \textbf{Viabilidad Técnica Comprobada, Adopción Limitada:} Estudios demuestran que ML/DL es técnicamente viable (30-50\% reducción en tiempo de codificación), pero adopción real enfrenta desafíos socio-técnicos: cambio de workflows, resistencia organizacional, integración con herramientas.

\item \textbf{Crisis de Reproducibilidad:} Liu et al. (2021) revela que solo 10.2\% de papers reportan reproducibilidad; 62.6\% no comparten código/datos. Limita confianza empresarial en tecnologías no validadas.

\item \textbf{Investigación Principalmente Académica:} Mayoría son case studies aislados o estudios teóricos. Falta investigación multiempresa y multisector.
\end{enumerate}

\subsection{1.4. Planteamiento del Problema de Investigación}

La literatura reciente (2021-2025) demuestra que Machine Learning y Generative AI son técnicamente viables para automatizar tareas en desarrollo de software. Sin embargo, existe un vacío significativo respecto a:

\begin{enumerate}
\item Cómo las organizaciones adoptan estas tecnologías de manera sostenible.
\item Cómo evoluciona el perfil profesional del ingeniero de software.
\item Qué competencias permanecen relevantes y cuáles se transforman.
\item Qué barreras organizacionales limitan la adopción real.
\end{enumerate}

\subsubsection{Relevancia:}

Profesionales en formación requieren comprender cómo prepararse para un mercado laboral transformado por IA. Organizaciones empresariales necesitan guía basada en evidencia para adoptar estas tecnologías de manera confiable y efectiva. Académicamente, esta investigación cierra la brecha entre viabilidad técnica y adopción organizacional.


\section*{References}

\begin{thebibliography}{00}

\bibitem{b1} S. Mustyala, P. Manchala, and M. Bisi, ``Case studies: Machine learning approaches for software development effort estimation,'' in \textit{Artificial Intelligence-Enhanced Software and Systems Engineering}, Springer, 2025.

\bibitem{b2} A. Rajbhoj, A. Somase, P. Kulkarni, and V. Kulkarni, ``Accelerating software development using generative AI: ChatGPT case study,'' in \textit{Proc. 17th Innovations in Software Engineering Conf. (ISEC)}, ACM, 2024.

\bibitem{b3} D. Russo, ``Navigating the complexity of generative AI adoption in software engineering,'' \textit{ACM Trans. Softw. Eng. Methodol.}, vol. 33, no. 5, 2024.

\bibitem{b4} A. Bahi, J. Gharib, and Y. Gahi, ``Integrating generative AI for advancing agile software development and mitigating project management challenges,'' \textit{Int. J. Adv. Comput. Sci. Appl.}, vol. 15, no. 3, 2024.

\bibitem{b5} C. Ebert and P. Louridas, ``Generative AI for software practitioners,'' \textit{IEEE Softw.}, vol. 40, no. 4, pp. 30--38, 2023.

\bibitem{b6} S. Wang, L. Huang, A. Gao, J. Ge, T. Zhang, H. Feng, I. Satyarth, M. Li, H. Zhang, and V. Ng, ``Machine/deep learning for software engineering: A systematic literature review,'' \textit{IEEE Trans. Softw. Eng.}, vol. 49, no. 3, pp. 1188--1231, 2023.

\bibitem{b7} H. Sofian, N. A. M. Yunus, and R. Ahmad, ``Artificial intelligence techniques in software engineering: A systematic mapping,'' \textit{IEEE Access}, vol. 10, pp. 51021--51040, 2022.

\bibitem{b8} C. Liu, C. Gao, X. Xia, D. Lo, J. Grundy, and X. Yang, ``On the reproducibility and replicability of deep learning in software engineering,'' \textit{ACM Trans. Softw. Eng. Methodol.}, vol. 31, no. 1, pp. 1--46, 2021.

\bibitem{b9} B. Hutchinson, A. Smart, A. Hanna, E. Denton, C. Greer, O. Kjartansson, P. Barnes, and M. Mitchell, ``Towards accountability for machine learning datasets: Practices from software engineering and infrastructure,'' in \textit{Proc. 2021 ACM Conf. Fairness, Accountability, and Transparency}, pp. 560--575, 2021.

\bibitem{b10} Y. Yang, X. Xia, D. Lo, and J. Grundy, ``A survey on deep learning for software engineering,'' \textit{ACM Comput. Surv.}, vol. 54, no. 10s, pp. 1--73, 2022.

\end{thebibliography}

\end{document}
